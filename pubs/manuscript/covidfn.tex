\documentclass[preprint, 3p,
authoryear]{elsarticle} %review=doublespace preprint=single 5p=2 column
%%% Begin My package additions %%%%%%%%%%%%%%%%%%%

\usepackage[hyphens]{url}

  \journal{Journal of Environmental Management} % Sets Journal name

\usepackage{graphicx}
%%%%%%%%%%%%%%%% end my additions to header

\usepackage[T1]{fontenc}
\usepackage{lmodern}
\usepackage{amssymb,amsmath}
% TODO: Currently lineno needs to be loaded after amsmath because of conflict
% https://github.com/latex-lineno/lineno/issues/5
\usepackage{lineno} % add
  \linenumbers % turns line numbering on
\usepackage{ifxetex,ifluatex}
\usepackage{fixltx2e} % provides \textsubscript
% use upquote if available, for straight quotes in verbatim environments
\IfFileExists{upquote.sty}{\usepackage{upquote}}{}
\ifnum 0\ifxetex 1\fi\ifluatex 1\fi=0 % if pdftex
  \usepackage[utf8]{inputenc}
\else % if luatex or xelatex
  \usepackage{fontspec}
  \ifxetex
    \usepackage{xltxtra,xunicode}
  \fi
  \defaultfontfeatures{Mapping=tex-text,Scale=MatchLowercase}
  \newcommand{\euro}{€}
\fi
% use microtype if available
\IfFileExists{microtype.sty}{\usepackage{microtype}}{}
\usepackage[]{natbib}
\bibliographystyle{plainnat}

\ifxetex
  \usepackage[setpagesize=false, % page size defined by xetex
              unicode=false, % unicode breaks when used with xetex
              xetex]{hyperref}
\else
  \usepackage[unicode=true]{hyperref}
\fi
\hypersetup{breaklinks=true,
            bookmarks=true,
            pdfauthor={},
            pdftitle={Indicators of social vulnerability explain rates of COVID-19 infections across Canada},
            colorlinks=false,
            urlcolor=blue,
            linkcolor=magenta,
            pdfborder={0 0 0}}

\setcounter{secnumdepth}{5}
% Pandoc toggle for numbering sections (defaults to be off)


% tightlist command for lists without linebreak
\providecommand{\tightlist}{%
  \setlength{\itemsep}{0pt}\setlength{\parskip}{0pt}}

% From pandoc table feature
\usepackage{longtable,booktabs,array}
\usepackage{calc} % for calculating minipage widths
% Correct order of tables after \paragraph or \subparagraph
\usepackage{etoolbox}
\makeatletter
\patchcmd\longtable{\par}{\if@noskipsec\mbox{}\fi\par}{}{}
\makeatother
% Allow footnotes in longtable head/foot
\IfFileExists{footnotehyper.sty}{\usepackage{footnotehyper}}{\usepackage{footnote}}
\makesavenoteenv{longtable}



\usepackage{setspace}
\usepackage{float}



\begin{document}


\begin{frontmatter}

  \title{Indicators of social vulnerability explain rates of COVID-19
infections across Canada}
    \author[uoft]{David Beauchesne%
  %
  }
   \ead{david.beauchesne@hotmail.com} 
    \author[tmu]{Eric Liberda%
  %
  }
   \ead{eric.liberda@torontomu.ca} 
    \author[tmu]{Robert Moriarity%
  %
  }
   \ead{rob.moriarity@torontomu.ca} 
    \author[uoft]{Nicholas Spence%
  %
  }
   \ead{nicholas.spence@utoronto.ca} 
    \author[uoft]{Leonard J.S. Tsuji%
  %
  }
   \ead{leonard.tsuji@utoronto.ca} 
    \author[qu]{Aleksandra Zuk%
  %
  }
   \ead{amz4@queensu.ca} 
      \affiliation[uoft]{
    organization={Department of Health and Society, University of
Toronto},addressline={27 King's College
Cir.},city={Toronto},postcode={M5S
1A1},state={Ontario},country={Canada},}
    \affiliation[tmu]{
    }
    \affiliation[qu]{
    }
    \cortext[cor1]{Corresponding author}
  
  \begin{abstract}
  Write abstract here in yaml header
  \end{abstract}
    \begin{keyword}
    COVID-19 \sep Social vulnerability \sep Canada \sep  \sep  \sep 
    
  \end{keyword}
  
 \end{frontmatter}

\onehalfspacing

\hypertarget{introduction}{%
\section{Introduction}\label{introduction}}

The objective of this article was to explore how rates of COVID-19
outcomes during the pandemic were linked to indicators of social
vulnerability (\emph{e.g.} employment, income, dwelling condition)
across Canada. Our specific goals were to 1) characterize the spatial
distribution of social vulnerabilities across Canada, and 2) explore the
relationship between social vulnerabilities and rates of COVID-19
infections and deaths during the pandemic. Our goal with this assessment
is identify readily available indicators of social vulnerability and
target those that should be alleviated to improve our response to
another pandemic in the future.

\hypertarget{material-and-methods}{%
\section{Material and methods}\label{material-and-methods}}

\hypertarget{spatial-scope}{%
\subsection{Spatial scope}\label{spatial-scope}}

This assessment is focused on a continental Canada-wide assessment
(Figure 1).

\hypertarget{data-sources}{%
\subsection{Data sources}\label{data-sources}}

\hypertarget{covid-19-data}{%
\subsubsection{COVID-19 data}\label{covid-19-data}}

\begin{quote}
The Timeline of COVID-19 in Canada (CovidTimelineCanada) is intended to
be the definitive source for data regarding the COVID-19 pandemic in
Canada. In addition to making available the ready-to-use datasets, this
repository also acts as a hub for collaboration on expanding and
improving the availability and quality of COVID-19 data in Canada. This
repository is maintained by the COVID-19 Canada Open Data Working Group
and is one component of the What Happened? COVID-19 in Canada project.
\end{quote}

This dataset provides daily data on COVID-19 infections and deaths for
99 Canadian health regions throughout the pandemic. We selected data
spanning from the onset of the pandemic in Canada (2020-01-15 in the
data) until 2023-05-17.

\textbf{\emph{Note that the end date should be discussed and decided
upon together. Some things to consider for this decision: when certain
provinces began phasing out reporting at the level of health regions
(see
https://github.com/ccodwg/CovidTimelineCanada/blob/main/docs/data\_sources/hr\_reporting.md),
and the advent of rapid testing.}}

\begin{longtable}[]{@{}lrrrr@{}}
\caption{Number of health regions, population size, and number of cases
and deaths for each Canadian province.}\tabularnewline
\toprule
Province & Health regions & Population & Cases & Deaths \\
\midrule
\endfirsthead
\toprule
Province & Health regions & Population & Cases & Deaths \\
\midrule
\endhead
AB & 5 & 4442879 & 632775 & 5733 \\
BC & 5 & 5214805 & 399103 & 5430 \\
MB & 5 & 1383765 & 156136 & 2501 \\
NB & 7 & 789225 & 90652 & 765 \\
NL & 4 & 520553 & 55335 & 344 \\
NS & 4 & 992055 & 85262 & 865 \\
NT & 1 & 45504 & 11508 & 22 \\
NU & 1 & 39403 & 3531 & 7 \\
ON & 34 & 14826276 & 1619039 & 16535 \\
PE & 1 & 164318 & 57251 & 103 \\
QC & 18 & 8604495 & 1336022 & 17858 \\
SK & 13 & 1204858 & 154058 & 1868 \\
YT & 1 & 42986 & 5588 & 32 \\
\bottomrule
\end{longtable}

\hypertarget{indicators-of-social-vulnerability}{%
\subsubsection{Indicators of social
vulnerability}\label{indicators-of-social-vulnerability}}

The whole country was divided into a 1 \(km^2\) grid cell to
characterize the selected indicators of social vulnerability. The list
of indicators used is available in Table 1. The following sections
present the different data sources and the indicators they were used to
obtain.

\textbf{\emph{Note that for a manuscript, I would focus on the
indicators of social vulnerability rather than the dataset, but to
facilitate our work, I present it here by focusing on datasets.}}

\begin{longtable}[]{@{}
  >{\raggedright\arraybackslash}p{(\columnwidth - 2\tabcolsep) * \real{0.68}}
  >{\raggedright\arraybackslash}p{(\columnwidth - 2\tabcolsep) * \real{0.32}}@{}}
\caption{List of indicators of social vulnerability considered for this
assessment.}\tabularnewline
\toprule
Indicators of social vulnerability & Source \\
\midrule
\endfirsthead
\toprule
Indicators of social vulnerability & Source \\
\midrule
\endhead
Gini index on adjusted household total income & 2021 Census of
Population \\
P90/P10 ratio on adjusted household after-tax income & 2021 Census of
Population \\
Prevalence of low income based on the Low-income measure, after tax
(LIM-AT) (\%) & 2021 Census of Population \\
Prevalence of low income based on the Low-income cut-offs, after tax
(LICO-AT) (\%) & 2021 Census of Population \\
Indigenous identity & 2021 Census of Population \\
Children in a one-parent family & 2021 Census of Population \\
Parents in one-parent families & 2021 Census of Population \\
No certificate, diploma or degree & 2021 Census of Population \\
Government transfers (\%) & 2021 Census of Population \\
Housing suitability & 2021 Census of Population \\
Dwelling condition & 2021 Census of Population \\
Acceptable housing & 2021 Census of Population \\
Distance to closest road & 2021 Census Road Network \\
Distance to closest critical healthcare facility & Open Database of
Healthcare Facilities \\
Distance to closest longterm healthcare facility & Open Database of
Healthcare Facilities \\
\bottomrule
\end{longtable}

\hypertarget{census-of-population}{%
\paragraph{2021 Census of Population}\label{census-of-population}}

Data from the 2021 Census of Population \citep{statisticscanada2021a}
was used to select relevant population indicators as proxies of social
vulnerabilities. The indicators were then joined to the census
cartographic divisions boundary files for 2021
\citep{statisticscanada2022f, statisticscanada2022} and subsequently
integrated in the study grid. The selected indicators were:

\begin{quote}
\begin{itemize}
\tightlist
\item
  \textbf{Gini index on adjusted household total income}: \emph{``The
  Gini coefficient is a number between zero and one that measures the
  relative degree of inequality in the distribution of income. The
  coefficient would register zero (minimum inequality) for a population
  in which each person received exactly the same adjusted household
  income and it would register a coefficient of one (maximum inequality)
  if one person received all the adjusted household income and the rest
  received none. Even though a single Gini coefficient value has no
  simple interpretation, comparisons of the level over time or between
  populations are very straightforward: the higher the coefficient, the
  higher the inequality of the distribution.''}
\end{itemize}
\end{quote}

\begin{quote}
\begin{itemize}
\tightlist
\item
  \textbf{P90/P10 ratio on adjusted household after-tax income}:
  \emph{``The P90/P10 ratio is a measure of inequality. It is the ratio
  of the 90th and the 10th percentile of the adjusted household
  after-tax income. The 90th percentile means 90\% of the population has
  income that falls below this threshold. The 10th percentile means 10\%
  of the population has income that falls below this threshold.''}
\end{itemize}
\end{quote}

\begin{quote}
\begin{itemize}
\tightlist
\item
  \textbf{Prevalence of low income based on the Low-income measure,
  after tax (LIM-AT) (\%)}: \emph{``The Low‑income measure, after tax,
  refers to a fixed percentage (50\%) of median adjusted after‑tax
  income of private households. The household after‑tax income is
  adjusted by an equivalence scale to take economies of scale into
  account. This adjustment for different household sizes reflects the
  fact that a household's needs increase, but at a decreasing rate, as
  the number of members increases.''}
\end{itemize}
\end{quote}

\begin{quote}
\begin{itemize}
\tightlist
\item
  \textbf{Prevalence of low income based on the Low-income cut-offs,
  after tax (LICO-AT) (\%)}: \emph{``The Low‑income cut‑offs, after tax
  refer to income thresholds, defined using 1992 expenditure data, below
  which economic families or persons not in economic families would
  likely have devoted a larger share of their after‑tax income than
  average to the necessities of food, shelter and clothing. More
  specifically, the thresholds represented income levels at which these
  families or persons were expected to spend 20 percentage points or
  more of their after‑tax income than average on food, shelter and
  clothing. These thresholds have been adjusted to current dollars using
  the all‑items Consumer Price Index (CPI).''}
\end{itemize}
\end{quote}

\begin{quote}
\begin{itemize}
\tightlist
\item
  \textbf{Indigenous identity}: \emph{``Indigenous identity refers to
  whether the person identified with the Indigenous peoples of Canada.
  This includes those who identify as First Nations (North American
  Indian), Métis and/or Inuk (Inuit), and/or those who report being
  Registered or Treaty Indians (that is, registered under the Indian Act
  of Canada), and/or those who have membership in a First Nation or
  Indian band. Aboriginal peoples of Canada (referred to here as
  Indigenous peoples) are defined in the Constitution Act, 1982, Section
  35 (2) as including the Indian, Inuit and Métis peoples of Canada.''}
\end{itemize}
\end{quote}

\begin{quote}
\begin{itemize}
\tightlist
\item
  \textbf{Children in a one-parent family}: \emph{``Percent children
  living in one-parent family.''}
\end{itemize}
\end{quote}

\begin{quote}
\begin{itemize}
\tightlist
\item
  \textbf{Parents in one-parent families}: \emph{``Percent parent in
  one-parent family,''}
\end{itemize}
\end{quote}

\begin{quote}
\begin{itemize}
\tightlist
\item
  \textbf{No certificate, diploma or degree}: \emph{``Percent population
  with no certificate, diploma or degree, population 25-64 years old.''}
\end{itemize}
\end{quote}

\begin{quote}
\begin{itemize}
\tightlist
\item
  \textbf{Government transfers (\%)}: \emph{``Percent of total income
  composed of government transfers in 2020, corresponding to all cash
  benefits received from federal, provincial, territorial or municipal
  governments during the reference period.''}
\end{itemize}
\end{quote}

Data from housing suitability \citep{statisticscanada2022c}, dwelling
condition \citep{statisticscanada2022d}, and acceptable housing
\citep{statisticscanada2022e} of the 2021 Census of Population
\citep{statisticscanada2021a} were also joined with the 2021 Census
cartographic division boundary file {[}\citet{statisticscanada2022};
statisticscanada2022f{]}.

\begin{quote}
\begin{itemize}
\tightlist
\item
  \textbf{Housing suitability}: According to Statistics Canada, housing
  suitability \emph{``refers to whether a private household is living in
  suitable accommodations according to the National Occupancy Standard
  (NOS); that is, whether the dwelling has enough bedrooms for the size
  and composition of the household. A household is deemed to be living
  in suitable accommodations if its dwelling has enough bedrooms, as
  calculated using the NOS. Housing suitability assesses the required
  number of bedrooms for a household based on the age, sex, and
  relationships among household members. An alternative variable,
  persons per room, considers all rooms in a private dwelling and the
  number of household members. Housing suitability and the National
  Occupancy Standard (NOS) on which it is based were developed by Canada
  Mortgage and Housing Corporation (CMHC) through consultations with
  provincial housing agencies.''} Housing suitability was assessed as
  the proportion of households in a census division considered as not
  suitable.
\end{itemize}
\end{quote}

\begin{quote}
\begin{itemize}
\tightlist
\item
  \textbf{Dwelling condition}: Dwelling condition refers to whether the
  dwelling is in need of repairs. Acceptability of dwelling condition
  was assessed as the proportion of households in a census division
  considered as needing major repairs.
\end{itemize}
\end{quote}

\begin{quote}
\begin{itemize}
\tightlist
\item
  \textbf{Acceptable housing}: According to Statistics Canada,
  acceptable housing \emph{``refers to whether a household meets each of
  the three indicator thresholds established by the Canada Mortgage and
  Housing Corporation for housing adequacy, suitability and
  affordability. Housing indicator thresholds are defined as follows: 1)
  adequate housing is reported by their residents as not requiring any
  major repairs; 2) affordable housing has shelter costs less than 30\%
  of total before-tax household income; 3) suitable housing has enough
  bedrooms for the size and composition of resident households according
  to the National Occupancy Standard (NOS), conceived by the Canada
  Mortgage and Housing Corporation and provincial and territorial
  representatives. Acceptable housing identifies which thresholds the
  household falls below, if any. Housing that is adequate in condition,
  suitable in size and affordable is considered to be acceptable.''}
  Here, acceptable housing was assessed as the proportion of households
  in a census division that was below any of the thresholds of adequacy,
  affordability or suitability.
\end{itemize}
\end{quote}

\hypertarget{census-2021-road-network-file}{%
\paragraph{Census 2021 road network
file}\label{census-2021-road-network-file}}

The road network from the Census 2021 road network file
{[}\citet{statisticscanada2021a}; statisticscanada2021b{]} was
rasterized as a 1 \(km^2\) resolution then integrated in the study grid
to obtain a raster of the distribution of the Canadian road network. The
distance of the centroid of each cell in the grid to the closest road
was then measured to obtain an assessment of the distance to the closest
road across Canada.

\hypertarget{open-database-of-healthcare-facilities}{%
\paragraph{Open Database of Healthcare
Facilities}\label{open-database-of-healthcare-facilities}}

The location of healthcare facilities available in the Open Database of
Healthcare Facilities {[}ODHF; \citet{statisticscanada2020};
\citet{statisticscanada2020a}{]} were used to assess the distance to the
closest healthcare facility. Facilities were divided between critical
and longterm care using the classifications available in the ODHF.
Hospitals and ambulatory health care services were considered as
critical care facilities, while nursing and residential care facilities
were considered as longterm care facilities.

\hypertarget{assessment}{%
\subsection{Assessment}\label{assessment}}

\hypertarget{integrated-dataset}{%
\subsubsection{Integrated dataset}\label{integrated-dataset}}

For each Canadian health unit, we extracted the number of cases, the
number of deaths, and the average value of each selected indicator of
social vulnerability.

\hypertarget{correlations}{%
\subsubsection{Correlations}\label{correlations}}

TODO: Assess the correlation between COVID-19 outcomes and indicators of
social vulnerability

\hypertarget{linear-regressions}{%
\subsubsection{Linear regressions}\label{linear-regressions}}

TODO: Build competing regression models to explain COVID-19 outcomes as
a function of social vulnerabilities. Perhaps use mixed models and use
provinces as a random factor in the assessment

\hypertarget{cluster-analysis}{%
\subsubsection{Cluster analysis}\label{cluster-analysis}}

TODO:

\begin{itemize}
\tightlist
\item
  Identify HR that are most similar in terms of COVID-19 outcomes and
  social indicators
\item
  Identify variables that explain intra-cluster similarity
\item
  Identify variables that explain inter-cluster dissimilarity
\end{itemize}

\hypertarget{results}{%
\section{Results}\label{results}}

\hypertarget{discussion}{%
\section{Discussion}\label{discussion}}

\hypertarget{credit-author-statememnt}{%
\section*{Credit author statememnt}\label{credit-author-statememnt}}
\addcontentsline{toc}{section}{Credit author statememnt}

\textbf{David Beauchesne}: Conceptualization, Methodology, Software,
Formal analysis, Data Curation, Writing - Original Draft, Writing --
review \& editing, Visualization; ****:

\hypertarget{declaration-of-competing-interest}{%
\section*{Declaration of competing
interest}\label{declaration-of-competing-interest}}
\addcontentsline{toc}{section}{Declaration of competing interest}

\hypertarget{acknowledgements}{%
\section*{Acknowledgements}\label{acknowledgements}}
\addcontentsline{toc}{section}{Acknowledgements}

\hypertarget{data-and-code-availability}{%
\section*{Data and code availability}\label{data-and-code-availability}}
\addcontentsline{toc}{section}{Data and code availability}

The data used for the cumulative effects assessment of marine shipping
in the St.~Lawrence and Saguenay Rivers are through Zenodo (DOI: ) and
available at the following link: \textbf{\emph{add link}}. The data
shared is the integration of all valued components and stressors within
our study grid, which is used to perform the cumulative effects
assessment. However, the data shared by First Nations is unavailable due
to data sharing agreements. The raw data are not available on Zenodo, as
the assessment integrates data from over 80 different projects. However,
code is available through a research compendiu (see below) to access all
publicly available data, which make up the vast majority of all data
used for the assessment. The code used for this assessment is available
through a research compendium called \emph{ceanav} and available through
a GitHub repository
(https://github.com/EffetsCumulatifsNavigation/ceanav) and archived on
Zenodo (DOI: ). \textbf{\emph{Note:}} \emph{DOIs will be added once
review process for publication is completed}

\newpage

\hypertarget{figures}{%
\section*{Figures}\label{figures}}
\addcontentsline{toc}{section}{Figures}

\hypertarget{tables}{%
\section*{Tables}\label{tables}}
\addcontentsline{toc}{section}{Tables}

\newpage

\renewcommand\refname{References}
\bibliography{covidfn.bib}


\end{document}
